% The Basics of Valkyrie OS
% A lightweight LaTeX skeleton to document the OS and its design.
\documentclass[11pt,a4paper]{article}
\usepackage[utf8]{inputenc}
\usepackage[T1]{fontenc}
\usepackage{lmodern}
\usepackage{geometry}
\geometry{margin=1in}
\usepackage{hyperref}
\hypersetup{
	colorlinks=true,
	linkcolor=blue,
	urlcolor=cyan,
}

% Code listings (works without external tools)
\usepackage{listings}
\usepackage{xcolor}
\lstdefinelanguage{asm86}{
	morekeywords={org,section,db,dd,global,extern,jmp,jz,jnz,call,ret,push,pop,mov,add,sub,cmp,jne,je},
	sensitive=true,
	morecomment=[l]{;}
}
\lstset{
	basicstyle=\ttfamily\small,
	keywordstyle=\color{teal}\bfseries,
	commentstyle=\color{gray}\itshape,
	numbers=left,
	numberstyle=\tiny\color{gray},
	stepnumber=1,
	numbersep=6pt,
	breaklines=true,
	breakatwhitespace=true,
	showstringspaces=false,
	tabsize=2,
}

% Shortcuts
\newcommand{\codefile}[2]{\lstinputlisting[language=#1]{#2}}

	\title{The Basics of Valkyrie OS}
\author{Vincent4486}
\date{\today}

\begin{document}
\maketitle
\begin{abstract}
This is the base guide to the Valkyrie Operating System. In this document would cover the code and 
features of Valkyrie OS and explains the code in the bootloader and kernel.
\end{abstract}

\clearpage

\section{Introduction}
The Valkyrie Operating System is made by Vincent4486 using C, other programming languages like Rust
or Go are expected to be used in the future.
\begin{itemize}
	\item Target: 32-bit x86 real-mode boot to a small kernel.
	\item Current status: Stage 1 and Stage 2 bootloaders exist; FAT driver lives in `src/bootloader/stage2`.
	\item Design: simple monolithic kernel written in C + assembly.
\end{itemize}

\section{Design goals}
List your goals here. Examples:
\begin{enumerate}
	\item Minimal, understandable code.
	\item Boot from FAT12/16 image.
	\item Small, well-documented boot stages.
	\item Incremental testing with tools in `tools/`.
\end{enumerate}

\section{Repository layout}
Explain the important folders and files so you don't have to remember them later.
\begin{description}
	\item[\texttt{src/}] Source code for the bootloader and kernel.
	\item[\texttt{src/bootloader/stage1/}] Minimal real-mode assembler stage 1.
	\item[\texttt{src/bootloader/stage2/}] C + assembly for the second stage; FAT, disk access and helpers.
	\item[\texttt{src/kernel/}] Kernel entrypoint and kernel-related assembly.
	\item[\texttt{tools/}] Utilities (e.g. FAT tools) useful during development.
	\item[\texttt{docs/}] This documentation.
\end{description}

\section{Boot process}
Describe how the system boots and which files are responsible for each step. Keep it chronological.

\subsection{Stage 1}
Explain the stage1 responsibilities (MBR/boot sector constraints) and point to the source file.
Example file: `src/bootloader/stage1/boot.asm`.

\subsection{Stage 2}
What stage2 does (switch to protected mode or load kernel), where the FAT driver is, and how you link it.
Key files: `src/bootloader/stage2/main.asm`, `src/bootloader/stage2/main.c`, `src/bootloader/stage2/fat.c`.

\section{Important code snippets}
You can include code directly or load files from the repo. When you build the PDF locally, listings will include the exact source.

Inline C example:
\begin{lstlisting}[language=C,caption={Minimal example from stdio.c}]
// src/bootloader/stage2/stdio.c
#include "stdio.h"
int puts(const char *s) {
	while (*s) putchar(*s++);
	return 0;
}
\end{lstlisting}

Inline assembly example:
\begin{lstlisting}[language=asm86,caption={Boot sector entry (example)}]
; src/bootloader/stage1/boot.asm
org 0x7c00
cli
xor ax, ax
mov ds, ax
; ...
\end{lstlisting}

To include an entire file from the tree (recommended for accuracy):
\begin{verbatim}
\codefile{C}{../src/bootloader/stage2/main.c}
\end{verbatim}

Replace the path above with the relative path from `docs/` to a source file.

\section{Build and run}
Keep simple reproducible commands here. From the project root (one level above `docs/`):
\begin{itemize}
	\item Build OS image: \texttt{make}
	\item Run (QEMU or emulator): \texttt{make run}
\end{itemize}

How to build this PDF (from `docs/`):
\begin{lstlisting}
# Recommended: use latexmk if available
latexmk -pdf -interaction=nonstopmode The_Basics_of_Valkyrie_OS.tex

# Or the simple sequence
pdflatex The_Basics_of_Valkyrie_OS.tex
bibtex The_Basics_of_Valkyrie_OS || true
pdflatex The_Basics_of_Valkyrie_OS.tex
pdflatex The_Basics_of_Valkyrie_OS.tex
\end{lstlisting}

Notes:
\begin{itemize}
	\item Using `minted` gives better highlighting but requires `-shell-escape` and Pygments; `listings` works everywhere.
	\item If you include large files from `src/` the PDF will reflect the current source at compile time.
\end{itemize}

\section{Development notes}
Keep short actionable items here: known bugs, TODOs, testing checklist, and where to find test images.

\section{Writing tips}
Short recommendations to keep docs usable:
\begin{itemize}
	\item Write short focused sections with a single purpose.
	\item Add relative paths when referencing source (so you can jump to the file from the PDF viewer in many editors).
	\item Use \verb|\label{}| and \verb|\ref{}| to cross-reference figures, sections, and listings.
	\item Prefer including source with \verb|\lstinputlisting| so examples stay correct.
\end{itemize}

\section{Appendix: Useful file list}
Add short one-line descriptions for the most important files so you don't have to hunt later.

\bigskip
\hrule
\vspace{6pt}
\noindent Last edited: \today

\end{document}

